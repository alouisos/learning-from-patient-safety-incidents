\glsaddall	
\printglossaries


\appendix


\chapter{Tools used}
\begin{itemize}
 \item SAS Excel plugin
 \item Python with NLTK, Brewery, Scikit-learn, and Matplotlib modules
 \item Carrot2
 \item RapidMiner with the text processing and named entity recognition plugins
 \item Apache Solr
 \item Google Refine
 \item Kile \LaTeX editor
 \item Jabref reference manager
 \item \LaTeX with bibtex, pythontex and several other packages
 \item Vim
 \item Ubuntu Linux
\end{itemize}

%\chapter{Data supplement}

%\section{XML files}
%file:///home/sam/computer-problems-clusters.xml
%file:///home/sam/bleep-prob.xml
%file:///home/sam/bleep.xml
%file:///home/sam/crash-prob.xml
%file:///home/sam/cerner-clusters.xml
%file:///home/sam/systemone-clusters.xml
%file:///home/sam/ongoingproblems.xml


%\section{XML style sheets}
%bleepprob.xsl  
%carrot2clusters.xsl   
%computers.xsl   
%example.xsl
%bleep.xsl       
%computerproblems.xsl  
%crashprob.xsl  
%solrresults.xsl


\chapter{Source code}

\section{Python code}
\label{pythoncode}

\subsection{lexicalanalysis.py}
\label{lexical_analysis.py}
\begin{pyverbatim}
#Adapted from code examples in S. Bird, E. Klein, and E. Loper. 
#Natural language processing with Python. O’Reilly Media, 2009
#Takes a text file and tokenizes it words, converts to lower case, 
#filters stop words, builds vocab for text, calculates lexical diversity, 
#builds collocation, builds frequency distribution of most common words, 
#builds example dispersion plot of words of interest 
#(manually entered below in this script), displays results 

#lexical_analysis.py


import nltk
from nltk.corpus import stopwords
from sys import argv

script, inputfilename = argv #takes whatever filename you pass in

print inputfilename

raw = open("%s" % inputfilename).read()


#loads incident descriptions identified as being due to computer problems

tokens = nltk.wordpunct_tokenize(raw) #tokenizes free text

#tokens = [nltk.PorterStemmer().stem(t) for t in tokens] 
# uncomment to stem tokens

#tokens = [nltk.WordNetLemmatizer().lemmatize(t) for t in tokens]
# uncomment to lemmatize tokens

text = nltk.Text(tokens) #defines text

words = [w.lower() for w in text] 
#defines words and makes all words lower case

filtered_words = [w for w in words if w not in stopwords.words('english')] 
#removes commonly occurring words ("stop words")

vocab = sorted(set(words)) #defines vocabulary

def lexical_diversity(text): #calculate lexical diversity
    return len(text) / len(set(words))

print "the number of words in the text is %d" % len(text)

print "the number of words in the vocabulary is %d" % len(vocab) 

print "lexical diversity is %d" % lexical_diversity(text) 
#prints lexical diversity

text.collocations() #builds collocations

fdist = nltk.FreqDist(ch.lower() for ch in filtered_words if ch.isalpha())

fdist.plot(50, cumulative=True) 
#prints a cumulative frequency distribution 
#of the 50 most commonly used words in the text

text.dispersion_plot(["computer", "system", "crash", "bleep", "patient"]) 
#example dispersion plot using arbitary search terms
\end{pyverbatim}

\subsection{mrchunk.py}
\label{mrchunk.py}
\begin{pyverbatim}
#takes a raw text file and extracts noun phrases to
#(hopefully) discover interesting relationships
#mrchunk.py

import nltk
import nltk.tag
from nltk.tokenize import word_tokenize, wordpunct_tokenize, sent_tokenize
from sys import argv


script, inputfilename = argv

raw = open("%s" % inputfilename).read()

sentences = sent_tokenize(raw)

wordsenttoke = [word_tokenize(t) for t in sent_tokenize(raw)]

chunks = nltk.tag.batch_pos_tag(wordsenttoke)

patterns =  "NP:{<DT>?<JJ>*<NN>}"
#DT = determiner e.g the (optional)
#JJ = adjective e.g big (can have any number of adjectives)
#NN = noun e.g dog

NPChunker = nltk.RegexpParser(patterns)

nplist = []


def sub_leaves(tree, node):
    return [t.leaves() for t in tree.subtrees
    (lambda s: s.node == node)]

#a tree traversal function for extracting NP chunks in the parsed tree
def chunk(patterns):
    for sent in chunks:
        tree = NPChunker.parse(sent)
        for subtree in tree.subtrees():
            if subtree.node == 'NP':
                print subtree
                nplist.append(subtree)
chunk(patterns)

print len(nplist)

ne_chunks = nltk.batch_ne_chunk(nplist) 

print ne_chunks #print all chunks

#print chunks tagged as people
for i in range(len(ne_chunks)):
    tree = ne_chunks[i]
    print 'PERSON'
    print sub_leaves(tree, 'PERSON')

#print chunks tagged as orgs
for i in range(len(ne_chunks)):
     tree = ne_chunks[i]
     print 'ORGANIZATION' 
     print sub_leaves(tree, 'ORGANIZATION')

#print chunks tagged as 
#geo-political entities 
for i in range(len(ne_chunks)):
     tree = ne_chunks[i]
     print 'GPE' 
     print sub_leaves(tree, 'GPE')

#fdist = nltk.FreqDist(nplist)
#fdist.plot(50, cumulative=True)

\end{pyverbatim}

\subsection{unableanalysis.py}
\label{unable_analysis.py}
\begin{pyverbatim}
#takes a text file and returns
#a frequency distribution plot of
#the 50 most common words to follow
#the expression ``unable to''
#unable_analysis.py

import nltk

raw = open('computers.csv').read()

tokens = nltk.wordpunct_tokenize(raw)

text = nltk.Text(tokens)

text.concordance("unable", width=40, lines=100)

text.findall("<unable><to><.*><.*>")

words = [w.lower() for w in text]

c = nltk.ConcordanceIndex(text.tokens)

unableset = [text.tokens[offset+2] for offset in c.offsets('unable')]

print len(unableset)

words = [w.lower() for w in unableset]

vocab = sorted(set(words))

print len(vocab)

fdist = nltk.FreqDist(unableset)

fdist.plot(50, cumulative=False)

fdist.items()[:10] #prints a list of the top 10

print vocab


\end{pyverbatim}

\subsection{csv2xml.py}
\label{csv2xml.py}
\begin{pyverbatim}
 #selects free text incident description column from .csv and
 #creates a Carrot2 .xml input file
 #csv2xml.py

 # -*- coding: utf-8 -*-
import sys,csv
from xml.dom.minidom import Document

def convert(input_file, output_file, header, query_text):
    print 'Reading: ', input_file
    print 'Writing: ', input_file
    print 'Data in: ', header

    doc = Document(  )
    results = doc.createElement("searchresult")
    doc.appendChild(results)
    query = doc.createElement("query")
    query.appendChild( doc.createTextNode(query_text) )
    results.appendChild( query )

    count = 0
    checked_header = False

    with open(input_file, 'Ur') as f:
        reader = csv.DictReader(f)
        for row_dict in reader:
            if not checked_header and header not in row_dict:
                print ("Unable to locate column named %s, check case and "
                         "header row in CSV" % header)
                sys.exit(1)
            checked_header = True

            results.appendChild( create_doc_element( doc, query_text,
                                                    row_dict[header], count ) )
            count = count + 1

    with open(output_file, "wb") as f:
        f.write( doc.toprettyxml(indent="  ", encoding="UTF-8") )

def create_doc_element(doc, title_text, text, id):
    n = doc.createElement('document')
    n.setAttribute("id", str(id))

    title = doc.createElement('title')
    title.appendChild( doc.createTextNode( title_text ) )

    url = doc.createElement('url')
    url.appendChild( doc.createTextNode( "http://localhost/" ) )

    snippet = doc.createElement('snippet')
    snippet.appendChild( doc.createTextNode( text ) )


    n.appendChild( title )
    n.appendChild( url )
    n.appendChild( snippet )
    return n

"""  <document id="0">
    <title>default</title>
    <url>http://www.globe.com.ph/</url>
    <snippet>
      Provides mobile communications (GSM) including
      GenTXT, handyphones, wireline services, an
      broadband Internet services.
    </snippet>
  </document>
  <document id="1">
    <title>Skate Shoes by Globe | Time For Change</title>
    <url>http://www.globeshoes.com/</url>
    <snippet>
      Skaters, surfers, and showboarders
      designing in their own style.
    </snippet>
  </document>

  ...

</searchresult>
"""

if __name__ == '__main__':
    if len(sys.argv) != 4:
        print """
Usage:
    python csv_to_xml.py <input file> <col name> <query text>

  i.e.
    python csv_to_xml.py input.xml IN07 "computer failures"

  You need to specify the input file and the column name, and don't
  forget to trim any junk before the header fields. Also don't forget to
  quote query text if it has spaces"""
        sys.exit(1)

    convert( sys.argv[1], sys.argv[1] + '.xml', sys.argv[2], sys.argv[3])
\end{pyverbatim}

\subsection{incidentsplit.py}
\label{incidentsplit.py}
\begin{pyverbatim}
 
#takes an input csv and splits it into a training.csv and testing.csv. 
#we do this to avoid overfitting when training classifiers.
#incidentsplit.py

import os
import csv

csvfile = csv.reader(open('computer_problems_IN07_and_category.csv', 'rb'))

trainingf = open('training.csv', 'wb')
testingf = open('testing.csv', 'wb')

trainingw = csv.writer(trainingf)
testingw = csv.writer(testingf)

#categories
#categories = ['train', 'test']

#set up a random index to split the files in 2:1 ratio
import numpy as np
from numpy.random import RandomState

n_samples = 7273 #number of rows in csv sample file

indices = np.arange(n_samples)
RandomState(42).shuffle(indices)
split = (n_samples * 2) / 3

#iterate through input file, split it, and write to two new files
for row in csvfile:
        if csvfile.line_num in indices[:split]:
            #category = 'train'
            trainingw.writerow(row)
        elif csvfile.line_num in indices[split:]: 
            #category = 'test'
            testingw.writerow(row)

trainingf.close()
testingf.close()


\end{pyverbatim}

\subsection{incidentextract.py}
\label{incidentextract.py}
\begin{pyverbatim}
#takes a csv and prints incident desc from row to pre-made folder
#matching severity of incident
#incidentextract.py

import os
import csv

#set up category directories
#dirname = 'category'
#if not os.path.exists(dirname):
#        os.makedirs(dirname)

#open csv file
csvfile = csv.reader(open('training.csv', 'rb')) #manually set this presently

#categories
#categories = ['Death', 'Severe', 'Moderate', 'Low', 'No Harm']

#set up row
i = 0

for row in csvfile:
        if row[1] == 'Death':
            category = 'Death' 
            #set incident report name 
            incident = category + str(i)
            print incident
            #set where will write incidents to
            completeName = os.path.abspath('./%s/%s' % (category, incident))
            #open file to write incident to
            f = open(completeName, 'w')
            #write to file
            f.write(row[0])
            #close file
            f.close()
            i = i + 1
        
        elif row[1] == 'Severe':
            category = 'Severe'
            incident = category + str(i)
            print incident
            completeName = os.path.abspath('./%s/%s' % (category, incident))
            f = open(completeName, 'w')
            f.write(row[0])
            f.close()
            i = i + 1

        elif row[1] == 'Moderate':
            category = 'Moderate'
            incident = category + str(i)
            print incident
            completeName = os.path.abspath('./%s/%s' % (category, incident))
            f = open(completeName, 'w')
            f.write(row[0])
            f.close()
            i = i + 1

        elif row[1] == 'Low':
            category = 'Low'
            incident = category + str(i)
            print incident
            completeName = os.path.abspath('./%s/%s' % (category, incident))
            f = open(completeName, 'w')
            f.write(row[0])
            f.close()
            i = i + 1

        elif row[1] == 'No Harm':
            category = 'NoHarm'
            incident = category + str(i)
            print incident
            completeName = os.path.abspath('./%s/%s' % (category, incident))
            f = open(completeName, 'w')
            f.write(row[0])
            f.close()
            i = i + 1
\end{pyverbatim}   

\subsection{classifierselection.py}
\label{classifierselection.py}
\begin{pyverbatim}
#20 newsgroup example adapted to use computer problem data
#classifierselection.py

"""
======================================================
Classification of text documents using sparse features
======================================================

This is an example showing how scikit-learn can be used to classify documents
by topics using a bag-of-words approach. This example uses a scipy.sparse
matrix to store the features and demonstrates various classifiers that can
efficiently handle sparse matrices.

The dataset used in this example is the 20 newsgroups dataset. It will be
automatically downloaded, then cached.

The bar plot indicates the accuracy, training time (normalized) and test time
(normalized) of each classifier.

"""

# Author: Peter Prettenhofer <peter.prettenhofer@gmail.com>
#         Olivier Grisel <olivier.grisel@ensta.org>
#         Mathieu Blondel <mathieu@mblondel.org>
#         Lars Buitinck <L.J.Buitinck@uva.nl>
# License: Simplified BSD

from __future__ import print_function

import logging
import numpy as np
from optparse import OptionParser
import sys
from time import time
import pylab as pl
import sklearn #so can use own data

from sklearn.datasets import fetch_20newsgroups 
from sklearn.feature_extraction.text import TfidfVectorizer
from sklearn.feature_extraction.text import HashingVectorizer
from sklearn.feature_selection import SelectKBest, chi2
from sklearn.linear_model import RidgeClassifier
from sklearn.svm import LinearSVC
from sklearn.linear_model import SGDClassifier
from sklearn.linear_model import Perceptron
from sklearn.linear_model import PassiveAggressiveClassifier
from sklearn.naive_bayes import BernoulliNB, MultinomialNB
from sklearn.neighbors import KNeighborsClassifier
from sklearn.neighbors import NearestCentroid
from sklearn.utils.extmath import density
from sklearn import metrics


# Display progress logs on stdout
logging.basicConfig(level=logging.INFO,
                    format='%(asctime)s %(levelname)s %(message)s')


# parse commandline arguments
op = OptionParser()
op.add_option("--report",
              action="store_true", dest="print_report",
              help="Print a detailed classification report.")
op.add_option("--chi2_select",
              action="store", type="int", dest="select_chi2",
              help="Select some number of features using a chi-squared test")
op.add_option("--confusion_matrix",
              action="store_true", dest="print_cm",
              help="Print the confusion matrix.")
op.add_option("--top10",
              action="store_true", dest="print_top10",
              help="Print ten most discriminative terms per class"
                   " for every classifier.")
op.add_option("--all_categories",
              action="store_true", dest="all_categories",
              help="Whether to use all categories or not.")
op.add_option("--use_hashing",
              action="store_true",
              help="Use a hashing vectorizer.")
op.add_option("--n_features",
              action="store", type=int, default=2 ** 16,
              help="n_features when using the hashing vectorizer.")


(opts, args) = op.parse_args()
if len(args) > 0:
    op.error("this script takes no arguments.")
    sys.exit(1)

print(__doc__)
op.print_help()
print()


###############################################################################
# Load some categories from the training set
if opts.all_categories:
    categories = None
else:
    categories = [
        'no harm',
        'low harm',
        'moderate harm',
        'severe harm',
        'death',
    ]

print("Loading patient safety incident dataset for categories:")
print(categories if categories else "all")

#data_train = fetch_20newsgroups(subset='train', categories=categories,
#                                shuffle=True, random_state=42)

#data_test = fetch_20newsgroups(subset='test', categories=categories,
#shuffle=True, random_state=42)

data_train = sklearn.datasets.load_files("training1balanced")
data_test = sklearn.datasets.load_files("testing1")


print('data loaded')

categories = data_train.target_names    # for case categories == None


def size_mb(docs):
    return sum(len(s.encode('utf-8')) for s in docs) / 1e6

data_train_size_mb = size_mb(data_train.data)
data_test_size_mb = size_mb(data_test.data)

print("%d documents - %0.3fMB (training set)" % (
    len(data_train.data), data_train_size_mb))
print("%d documents - %0.3fMB (training set)" % (
    len(data_test.data), data_test_size_mb))
print("%d categories" % len(categories))
print()

# split a training set and a test set
y_train, y_test = data_train.target, data_test.target

print("Extracting features from the training dataset using a sparse vectorizer")
t0 = time()
if opts.use_hashing:
    vectorizer = HashingVectorizer(stop_words='english', non_negative=True,
                                   n_features=opts.n_features)
    X_train = vectorizer.transform(data_train.data)
else:
    vectorizer = TfidfVectorizer(sublinear_tf=True, max_df=0.5,
                                 stop_words='english')
    X_train = vectorizer.fit_transform(data_train.data)
duration = time() - t0
print("done in %fs at %0.3fMB/s" % (duration, data_train_size_mb / duration))
print("n_samples: %d, n_features: %d" % X_train.shape)
print()

print("Extracting features from the test dataset using the same vectorizer")
t0 = time()
X_test = vectorizer.transform(data_test.data)
duration = time() - t0
print("done in %fs at %0.3fMB/s" % (duration, data_test_size_mb / duration))
print("n_samples: %d, n_features: %d" % X_test.shape)
print()

if opts.select_chi2:
    print("Extracting %d best features by a chi-squared test" %
          opts.select_chi2)
    t0 = time()
    ch2 = SelectKBest(chi2, k=opts.select_chi2)
    X_train = ch2.fit_transform(X_train, y_train)
    X_test = ch2.transform(X_test)
    print("done in %fs" % (time() - t0))
    print()


def trim(s):
    """Trim string to fit on terminal (assuming 80-column display)"""
    return s if len(s) <= 80 else s[:77] + "..."


# mapping from integer feature name to original token string
if opts.use_hashing:
    feature_names = None
else:
    feature_names = np.asarray(vectorizer.get_feature_names())


###############################################################################
# Benchmark classifiers
def benchmark(clf):
    print('_' * 80)
    print("Training: ")
    print(clf)
    t0 = time()
    clf.fit(X_train, y_train)
    train_time = time() - t0
    print("train time: %0.3fs" % train_time)

    t0 = time()
    pred = clf.predict(X_test)
    test_time = time() - t0
    print("test time:  %0.3fs" % test_time)

    score = metrics.f1_score(y_test, pred)
    print("f1-score:   %0.3f" % score)

    if hasattr(clf, 'coef_'):
#        print("dimensionality: %d" % clf.coef_.shape[1])
        print("density: %f" % density(clf.coef_))

        if opts.print_top10 and feature_names is not None:
            print("top 10 keywords per class:")
            for i, category in enumerate(categories):
                top10 = np.argsort(clf.coef_[i])[-10:]
                print(trim("%s: %s"
                      % (category, " ".join(feature_names[top10]))))
        print()

    if opts.print_report:
        print("classification report:")
        print(metrics.classification_report(y_test, pred,
                                            target_names=categories))

    if opts.print_cm:
        print("confusion matrix:")
        print(metrics.confusion_matrix(y_test, pred))

    print()
    clf_descr = str(clf).split('(')[0]
    return clf_descr, score, train_time, test_time


results = []
for clf, name in (
        (RidgeClassifier(tol=1e-2, solver="lsqr"), "Ridge Classifier"),
        (Perceptron(n_iter=50), "Perceptron"),
        (PassiveAggressiveClassifier(n_iter=50), "Passive-Aggressive"),
        (KNeighborsClassifier(n_neighbors=10), "kNN")):
    print('=' * 80)
    print(name)
    results.append(benchmark(clf))

for penalty in ["l2", "l1"]:
    print('=' * 80)
    print("%s penalty" % penalty.upper())
    # Train Liblinear model
    results.append(benchmark(LinearSVC(loss='l2', penalty=penalty,
                                            dual=False, tol=1e-3)))

    # Train SGD model
    results.append(benchmark(SGDClassifier(alpha=.0001, n_iter=50,
                                           penalty=penalty)))

# Train SGD with Elastic Net penalty
print('=' * 80)
print("Elastic-Net penalty")
results.append(benchmark(SGDClassifier(alpha=.0001, n_iter=50,
                                       penalty="elasticnet")))

# Train NearestCentroid without threshold
print('=' * 80)
print("NearestCentroid (aka Rocchio classifier)")
results.append(benchmark(NearestCentroid()))

# Train sparse Naive Bayes classifiers
print('=' * 80)
print("Naive Bayes")
results.append(benchmark(MultinomialNB(alpha=.01)))
results.append(benchmark(BernoulliNB(alpha=.01)))


class L1LinearSVC(LinearSVC):

    def fit(self, X, y):
        # The smaller C, the stronger the regularization.
        # The more regularization, the more sparsity.
        self.transformer_ = LinearSVC(penalty="l1",
                                      dual=False, tol=1e-3)
        X = self.transformer_.fit_transform(X, y)
        return LinearSVC.fit(self, X, y)

    def predict(self, X):
        X = self.transformer_.transform(X)
        return LinearSVC.predict(self, X)

print('=' * 80)
print("LinearSVC with L1-based feature selection")
results.append(benchmark(L1LinearSVC()))


# make some plots

indices = np.arange(len(results))

results = [[x[i] for x in results] for i in range(4)]

clf_names, score, training_time, test_time = results
training_time = np.array(training_time) / np.max(training_time)
test_time = np.array(test_time) / np.max(test_time)

pl.title("Score")
pl.barh(indices, score, .2, label="score", color='r')
pl.barh(indices + .3, training_time, .2, label="training time", color='g')
pl.barh(indices + .6, test_time, .2, label="test time", color='b')
pl.yticks(())
pl.legend(loc='best')
pl.subplots_adjust(left=.25)

for i, c in zip(indices, clf_names):
    pl.text(-.3, i, c)

pl.show()
\end{pyverbatim}

\subsection{randomselect.py}
\label{randomselect.py}
\begin{pyverbatim}
#select random sample of incidents for balancing
#randomselect.py

import os
import random

#open text file containing items that aren't deaths 
#and severes from the training set
listfile = open('list.txt', 'rb') 

#load items (incident files)in text file as a list
allitems = [row for row in listfile] 

mylist = allitems #call it mylist

rand_smpl = [ mylist[i] for i in sorted(random.sample(xrange(len(mylist)), 51)) ] 
#select a random sample of 51 incidents from the list 
#(because we're balancing with incidents that are deaths and severes and 
#there are 51 deaths and severes in the training set)

#open a file to write our random selection of 51 incidents to
f = open('randlist.txt', 'w') 
for item in rand_smpl:
     print>>f, item
    
f.close()
\end{pyverbatim}

\subsection{incidentclassify.py}
\label{incident_classify.py}
\begin{pyverbatim}
#incident severity classifier
#incidenclassify.py


import sklearn
from sklearn import *

from nltk import word_tokenize          
from nltk.stem import WordNetLemmatizer 
class LemmaTokenizer(object):
    def __init__(self):
         self.wnl = WordNetLemmatizer()
    def __call__(self, doc):
         return [self.wnl.lemmatize(t) for t in word_tokenize(doc)]


#load incidents from folder called container which contains incidents 
#in appropriate category subfolders into an sklearn "bunch"
computer_incidents_training = sklearn.datasets.load_files("training")
computer_incidents_testing = sklearn.datasets.load_files("testing")

#CountVectorizer converts a collection of text documents into a matrix 
#of token counts. By default words are tokenized.

from sklearn.feature_extraction.text import CountVectorizer
count_vect = CountVectorizer()
X_counts = count_vect.fit_transform(computer_incidents_training.data)

#term occurances (X_counts) fails to take account of document size. 
#Longer documents may have higher average count values than shorter 
#ones even though they talk about the same topic. 
#To avoid this the number of occurrences in any document is divided 
#by the total number of words in that document to give the 
#Term Frequency (tf).

from sklearn.feature_extraction.text import TfidfTransformer
tf_transformer = TfidfTransformer(use_idf=False).fit(X_counts)
X_counts_tf = tf_transformer.transform(X_counts)

#a further refinement is to downscale the weighting of words that 
#occur in many documents in the corpus and are therefore less 
#informative than words only occurring in a small portion of the corpus. 
#This is results in the term frequency inverse document frequency 
#measure (tf-idf).

tfidf_transformer = TfidfTransformer()
X_counts_tfidf = tf_transformer.fit_transform(X_counts)
#print X_counts_tfidf.shape for error checking, 
#prints number of samples and number of features

print X_counts_tfidf.shape

#ready train up a classifier and evaluate performance :-)

from sklearn.naive_bayes import MultinomialNB
clf = MultinomialNB().fit(X_counts_tfidf, computer_incidents_training.target)

#test case with an incident
docs_new = ["""Main A / E Radioogy Viewing Area Computers are soooo slow , 
and getting slowing throughout the day . Impacting on patient through - put 
and accurate recording of patient exam times ."""]

X_new_counts = count_vect.transform(docs_new)
X_new_tfidf = tfidf_transformer.fit_transform(X_new_counts)

predicted = clf.predict(X_new_tfidf)

for doc, category in zip(docs_new, predicted):
    print '%r => %s' % (doc, computer_incidents_training.target_names[category])

#test case using a pipe
from sklearn.pipeline import Pipeline
text_clf = Pipeline([
    ('vect', CountVectorizer()),
    ('tfidf', TfidfTransformer()),
    ('clf', MultinomialNB()),
    ])

_ = text_clf.fit(computer_incidents_training.data, computer_incidents_training.target)

predicted = _.predict(docs_new)
for doc, category in zip(docs_new, predicted):
        print '%r => %s' % (doc, computer_incidents_training.target_names[category])

import numpy as np

docs_test = computer_incidents_testing.data
predicted = text_clf.predict(docs_test)
print np.mean(predicted == computer_incidents_testing.target) 
#prints accuracy of the model

from sklearn import metrics
print metrics.classification_report(
    computer_incidents_testing.target, predicted,
    target_names = computer_incidents_testing.target_names) 
#prints precision, f1 and support of the model



from sklearn.linear_model import SGDClassifier
text_clf = Pipeline([
     ('vect', CountVectorizer()),
     ('tfidf', TfidfTransformer()),
     ('clf', SGDClassifier(loss='hinge', penalty='l2',
                           alpha=1e-3, n_iter=5)),
])
_ = text_clf.fit(computer_incidents_training.data, computer_incidents_training.target)
predicted = text_clf.predict(docs_test)
print np.mean(predicted == computer_incidents_testing.target)
#prints accuracy of the model

from sklearn import metrics
print metrics.classification_report(
    computer_incidents_testing.target, predicted,
    target_names = computer_incidents_testing.target_names) 
#prints precision, f1 and support of the model


print metrics.confusion_matrix(computer_incidents_testing.target, predicted)

#optimize parameters from grid search
from sklearn.linear_model import SGDClassifier
text_clf = Pipeline([
     ('vect', CountVectorizer(max_df=1.0, max_features=None, ngram_range=(1, 2))),
     ('tfidf', TfidfTransformer(norm='l1', use_idf=True )),
     ('clf', SGDClassifier(loss='hinge', penalty='l2',
                          alpha=1e-05, n_iter=80)),
])
_ = text_clf.fit(computer_incidents_training.data, computer_incidents_training.target)
predicted = text_clf.predict(docs_test)
print np.mean(predicted == computer_incidents_testing.target)#prints accuracy of the model

from sklearn import metrics
print metrics.classification_report(
    computer_incidents_testing.target, predicted,
    target_names = computer_incidents_testing.target_names) 
#prints precision, f1 and support of the model

predicted = text_clf.predict(docs_test)

print np.mean(predicted == computer_incidents_testing.target)
#prints accuracy of the model

print metrics.confusion_matrix(computer_incidents_testing.target, predicted)

#optimize parameters from grid search
from sklearn.linear_model import SGDClassifier
text_clf = Pipeline([
     ('vect', CountVectorizer(max_df=1.0, max_features=None, ngram_range=(1, 2), tokenizer=LemmaTokenizer())),
     ('tfidf', TfidfTransformer(norm='l1', use_idf=True )),
     ('clf', SGDClassifier(loss='hinge', penalty='l2',
                           alpha=1e-05, n_iter=80)),
])
_ = text_clf.fit(computer_incidents_training.data, computer_incidents_training.target)
predicted = text_clf.predict(docs_test)
print np.mean(predicted == computer_incidents_testing.target)
#prints accuracy of the model

from sklearn import metrics
print metrics.classification_report(computer_incidents_testing.target, predicted,
target_names = computer_incidents_testing.target_names) 
#prints precision, f1 and support of the model

predicted = text_clf.predict(docs_test)

print np.mean(predicted == computer_incidents_testing.target)
#prints accuracy of the model

print metrics.confusion_matrix(computer_incidents_testing.target, predicted)

#optimize parameters from grid search
from sklearn.linear_model import SGDClassifier
text_clf = Pipeline([
     ('vect', CountVectorizer(max_df=1.0, max_features=None, ngram_range=(1, 2), tokenizer=LemmaTokenizer())),
     ('tfidf', TfidfTransformer(norm='l1', use_idf=True )),
     ('clf', SGDClassifier(loss='hinge', penalty='l2',
                           alpha=1e-05, n_iter=80, class_weight='auto')),
])
_ = text_clf.fit(computer_incidents_training.data, computer_incidents_training.target)
predicted = text_clf.predict(docs_test)
print np.mean(predicted == computer_incidents_testing.target)
#prints accuracy of the model

from sklearn import metrics
print metrics.classification_report(computer_incidents_testing.target, predicted,
target_names = computer_incidents_testing.target_names) 
#prints precision, f1 and support of the model

predicted = text_clf.predict(docs_test)

print np.mean(predicted == computer_incidents_testing.target)
#prints accuracy of the model

print metrics.confusion_matrix(computer_incidents_testing.target, predicted)

\end{pyverbatim}

\subsection{gridsearch.py}
\label{gridsearch.py}
\begin{pyverbatim}
 
#grid search example adapted to run on computer problem data
#gridsearch.py

"""
==========================================================
Sample pipeline for text feature extraction and evaluation
==========================================================

The dataset used in this example is the 20 newsgroups dataset which will be
automatically downloaded and then cached and reused for the document
classification example.

You can adjust the number of categories by giving there name to the dataset
loader or setting them to None to get the 20 of them.

Here is a sample output of a run on a quad-core machine::

  Loading 20 newsgroups dataset for categories:
  ['alt.atheism', 'talk.religion.misc']
  1427 documents
  2 categories

  Performing grid search...
  pipeline: ['vect', 'tfidf', 'clf']
  parameters:
  {'clf__alpha': (1.0000000000000001e-05, 9.9999999999999995e-07),
   'clf__n_iter': (10, 50, 80),
   'clf__penalty': ('l2', 'elasticnet'),
   'tfidf__use_idf': (True, False),
   'vect__max_n': (1, 2),
   'vect__max_df': (0.5, 0.75, 1.0),
   'vect__max_features': (None, 5000, 10000, 50000)}
  done in 1737.030s

  Best score: 0.940
  Best parameters set:
      clf__alpha: 9.9999999999999995e-07
      clf__n_iter: 50
      clf__penalty: 'elasticnet'
      tfidf__use_idf: True
      vect__max_n: 2
      vect__max_df: 0.75
      vect__max_features: 50000

"""
print __doc__

# Author: Olivier Grisel <olivier.grisel@ensta.org>
#         Peter Prettenhofer <peter.prettenhofer@gmail.com>
#         Mathieu Blondel <mathieu@mblondel.org>
# License: Simplified BSD

from pprint import pprint
from time import time
import logging
import sklearn

from sklearn.datasets import fetch_20newsgroups
from sklearn.datasets import load_files #allow to use my data
from sklearn.feature_extraction.text import CountVectorizer
from sklearn.feature_extraction.text import TfidfTransformer
from sklearn.linear_model import SGDClassifier
from sklearn.grid_search import GridSearchCV
from sklearn.pipeline import Pipeline

# Display progress logs on stdout
logging.basicConfig(level=logging.INFO,
                    format='%(asctime)s %(levelname)s %(message)s')


###############################################################################
# Load some categories from the training set
categories = [
    'alt.atheism',
    'talk.religion.misc',
]
# Uncomment the following to do the analysis on all the categories
#categories = None

print "Loading 20 newsgroups dataset for categories:"
print categories

#data = fetch_20newsgroups(subset='train', categories=categories)
data = sklearn.datasets.load_files("training") #make read my data

print "%d documents" % len(data.filenames)
print "%d categories" % len(data.target_names)
print

###############################################################################
# define a pipeline combining a text feature extractor with a simple
# classifier
pipeline = Pipeline([
    ('vect', CountVectorizer()),
    ('tfidf', TfidfTransformer()),
    ('clf', SGDClassifier()),
])

parameters = {
    # uncommenting more parameters will give better exploring power but will
    # increase processing time in a combinatorial way
    'vect__max_df': (0.5, 0.75, 1.0),
    'vect__max_features': (None, 5000, 10000, 50000),
    'vect__ngram_range': ((1, 1), (1, 2)),  # unigrams or bigrams
    'tfidf__use_idf': (True, False),
    'tfidf__norm': ('l1', 'l2'),
    'clf__alpha': (0.00001, 0.000001),
    'clf__penalty': ('l2', 'elasticnet'),
    'clf__n_iter': (10, 50, 80),
}

if __name__ == "__main__":
    # multiprocessing requires the fork to happen in a __main__ protected
    # block

    # find the best parameters for both the feature extraction and the
    # classifier
    grid_search = GridSearchCV(pipeline, parameters, n_jobs=-1, verbose=1)

    print "Performing grid search..."
    print "pipeline:", [name for name, _ in pipeline.steps]
    print "parameters:"
    pprint(parameters)
    t0 = time()
    grid_search.fit(data.data, data.target)
    print "done in %0.3fs" % (time() - t0)
    print

    print "Best score: %0.3f" % grid_search.best_score_
    print "Best parameters set:"
    best_parameters = grid_search.best_estimator_.get_params()
    for param_name in sorted(parameters.keys()):
        print "\t%s: %r" % (param_name, best_parameters[param_name])

\end{pyverbatim}

\subsection{Python brewery script}
\label{brewery_script.py}
\begin{pyverbatim}
 #audits csv file for field completion
 #brewery_script.py

import brewery
from sys import argv

script, inputfilename = argv

b = brewery.create_builder()
b.csv_source("%s" % inputfilename)
b.audit()

b.pretty_printer()

b.stream.run()

\end{pyverbatim}

%\begin{minted}{python}
%import brewery

%b = brewery.create_builder()
%b.csv_source("computer_problems.csv")
%b.audit()

%f = b.fork()
%f.csv_target("computer_problems_audit.csv")

%b.pretty_printer()
%b.stream.run()      
%\end{minted}

%\subsection{Lexical analysis with NLTK}
%\begin{itemize}
% \item lexicalanalysis.py 
% \item unableanalysis.py
% \item graphtest.py       
%\end{itemize}

%\subsection{Collocations}
%collocation.py 

%\subsection{Chunking}  
%\begin{itemize}
% \item chunkanalysis.py  
% \item chunking.py     
% \item chunking2.py        
% \item chunking3.py 
% \item chunking4.py      
%\end{itemize}
     

%\subsection{POS tagging}
%\begin{itemize}
% \item postagging.py
% \item postagging2.py
%\end{itemize}


%\subsection{File preparation for Carrot2}
%csvtoxml.py

%\chapter{Solr configuration files}
%\label{solrconfig}
%provided separately
