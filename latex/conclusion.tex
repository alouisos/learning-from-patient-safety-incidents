\textbf{\textit{``Knowing is not enough; we must apply. Willing is not enough; we must do.''}}
Johann Wolfgang von Goethe (1749-1832)

\chapter{Conclusion}

\section{Can data mining help us to learn from patient safety incidents?}

\subsection{Yes for problem topic discovery}
Data mining can demonstrably help with problem discovery in a non-trivial way. Some of the candidate clusters and cluster-labels generated by the lingo algorithm are meaningful and useful. The general purpose tools and techniques of data mining have the potential to allow appropriately trained analysts to carry out their role more efficiently, and may improve novel incident type detection.

\subsection{Probably for specific classification tasks}
There is ample evidence from the literature that accurate predictive models can be built for classification purposes. In some instances, such as free text severity classification, these methods could have immediate utility.

Other applications, such as classifying circumstances and acting to prevent harm in an automated fashion, or facilitating strategies to prevent harm, using electronic medical records, will be important, and helpful, applications of data mining in the future.

With the increasing digitisation of health care demonstration of utility may be more easily obtained when reporting, expert analysis, and action are more closely integrated into tighter feedback loops. For example a classifier used to predict prescription error might be constructed and tied to a prevention intervention with the goal of reducing the rate of a specific measurable harm occurring. 

For the subset of patient safety incidents relating to computer use considered, the application of DM methods suggests that the NRLS system does not result in the timely resolution of safety issues. For safety incidents due to computer problems, and possibly other types of safety incident, lessons might be learned from a more general approaches to, and cultures of, systems improvement. In particular, the more open and action focussed approach to improving quality present in bug reporting systems found in the open source software community has intuitive appeal.

