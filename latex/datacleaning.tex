\section{Data cleaning}

The source data for ``computer problem'' extract are incident reports which are completed by individual health care professionals and submitted electronically. Duplicate report submission and submission of reports with missing data is possible and it was therefore important to deduplicate the data and establish its quality.

\section{Deduplication}
google-refine\cite{huynhgoogle} was used to remove duplicates by faceting on the incident description free text field (IN07). 

\section{Reconciliation}
Reconciliation was not carried out for time reasons although google-refine\cite{huynhgoogle} does support this. This would have been helpful for correcting spellings and handling acronyms and synonyms.

\section{Establishing data quality}
google-refine\cite{huynhgoogle} was used to establish median field length and number of words per field using the Google Refine Expression Language (GREL). The Python Brewery\cite{pythonbrewery} module was used to calculate field completeness for the data set:\ref{breweryscript}

%\begin{minted}{python}
import brewery
from sys import argv

script, inputfilename = argv

b = brewery.create_builder()
b.csv_source("%s" % inputfilename)
b.audit()

b.pretty_printer()

b.stream.run()
\end{minted}

\begin{minted}{python}
#Adapted from code examples in S. Bird, E. Klein, and E. Loper. 
#Natural language processing with Python. O’Reilly Media, 2009
#
#Takes a text file and tokenizes it words, converts to lower lase, filters stop words, 
#builds vocab for text, calculates lexical diversity, builds collocation, builds frequency 
#distribtion of most common words, builds example dispersion plot of words of interest 
#(manually entered below in this script), then displays results 

import nltk
from nltk.corpus import stopwords
from sys import argv

script, inputfilename = argv #takes whatever filename you pass in

print inputfilename

raw = open("%s" % inputfilename).read() #loads incident descriptions identified as being due to computer problems

tokens = nltk.wordpunct_tokenize(raw) #tokenizes free text

#tokens = [nltk.PorterStemmer().stem(t) for t in tokens] 
# uncomment to stem tokens

#tokens = [nltk.WordNetLemmatizer().lemmatize(t) for t in tokens]
# uncomment to lemmatize tokens

text = nltk.Text(tokens) #defines text

words = [w.lower() for w in text] #defines words and makes all words lower case

filtered_words = [w for w in words if not w in stopwords.words('english')] #removes commonly occuring words ("stop words")

vocab = sorted(set(words)) #defines vocabulary

def lexical_diversity(text): #calculate lexical diversity
    return len(text) / len(set(words))

print "the number of words in the text is %d" % len(text)

print "the number of words in the vocabulary is %d" % len(vocab) 

print "lexical diversity is %d" % lexical_diversity(text) #prints lexical diversity

text.collocations() #builds collocations

fdist = nltk.FreqDist(filtered_words)

fdist.plot(50, cumulative=True) #prints a cumulative frequency distribution of the 50 most commonly used words in the text

text.dispersion_plot(["computer", "system", "crash", "bleep", "patient"]) #example dispersion plot using arbitary seach terms
\end{minted}




%\section detecting disguised missing data %\cite{Belen2011}

