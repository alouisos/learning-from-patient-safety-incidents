%graphics
\usepackage{graphicx}
\usepackage{subfigure}
\usepackage{pslatex}
\usepackage{pstricks}

%math equations
\usepackage{amsmath}

%python code
%\usepackage{minted}

%headers
\usepackage{fancyhdr}
\setlength{\headheight}{15.2pt}
%\pagestyle{fancy} is set in details.tex because I wish headers to begin after contents

%display URLS
\usepackage{url}

%hyperlinks
\usepackage{hyperref}

%comments
\usepackage{verbatim} 

%nice tables
\usepackage{booktabs}
\newcommand{\ra}[1]{\renewcommand{\arraystretch}{#1}}

%multi rows for the nice tables
\usepackage{multirow} 

%nice diplay of code
%\usepackage{minted}

%nice references
\usepackage[super]{natbib}

%some maths
\usepackage{amsmath}

%line spacing
\renewcommand{\baselinestretch}{1.5} 

%margins
\usepackage{geometry}
\geometry{verbose,a4paper,tmargin=25mm,bmargin=25mm,lmargin=40mm,rmargin=25mm}

%glossary
\usepackage[toc]{glossaries}
\makeglossary

%pythontex
%pythontex
% Engine-specific settings
% Detect pdftex/xetex/luatex, and load appropriate font packages.
% This is inspired by the approach in the iftex package.
% pdftex:
\expandafter\ifx\csname pdfmatch\endcsname\relax
\else
    \usepackage[T1]{fontenc}
    \usepackage[utf8]{inputenc}
\fi
% xetex:
\expandafter\ifx\csname XeTeXinterchartoks\endcsname\relax
\else
    \usepackage{fontspec}
    \defaultfontfeatures{Ligatures=TeX}
\fi
% luatex:
\expandafter\ifx\csname directlua\endcsname\relax
\else
    \usepackage{fontspec}
\fi
% End engine-specific settings


\usepackage[makestderr]{pythontex}
\restartpythontexsession{\thesection}


\usepackage[framemethod=TikZ]{mdframed}

\newcommand{\pytex}{Python\TeX}
\renewcommand*{\thefootnote}{\fnsymbol{footnote}}

%misc
%\def\naive{na\"\i ve}
%\def\Naive{Na\"\i ve}

%misc
%\def\naive{na\"\i ve}
%\def\Naive{Na\"\i ve}
